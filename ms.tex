%%
%% This is file `template-8s.tex',
%% generated with the docstrip utility.
%%
%% The original source files were:
%%
%% template.raw  (with options: `8s')
%% 
%% Template for the LaTeX class aipproc.
%% 
%% (C) 1998,2000,2001 American Institute of Physics and Frank Mittelbach
%% All rights reserved
%% 
%%
%% $Id: template.raw,v 1.12 2005/07/06 19:22:14 frank Exp $
%%

%%%%%%%%%%%%%%%%%%%%%%%%%%%%%%%%%%%%%%%%%%%%
%% Please remove the next line of code if you
%% are satisfied that your installation is
%% complete and working.
%%
%% It is only there to help you in detecting
%% potential problems.
%%%%%%%%%%%%%%%%%%%%%%%%%%%%%%%%%%%%%%%%%%%%

\input{aipcheck}

%%%%%%%%%%%%%%%%%%%%%%%%%%%%%%%%%%%%%%%%%%%%
%% SELECT THE LAYOUT
%%
%% The class supports further options.
%% See aipguide.pdf for details.
%%
%%%%%%%%%%%%%%%%%%%%%%%%%%%%%%%%%%%%%%%%%%%%

\documentclass[
  ,final            % use final for the camera ready runs
%%    ,draft            % use draft while you are working on the paper
%%    ,numberedheadings % uncomment this option for numbered sections
%%  ,                 % add further options here if necessary
  ]
  {aipproc}

\layoutstyle{8x11single}
\usepackage{amsmath}
\usepackage{dsfont}

\newcommand{\pars}{\boldsymbol{\theta}}
\newcommand{\data}{\mathbf{x}}

%%%%%%%%%%%%%%%%%%%%%%%%%%%%%%%%%%%%%%%%%%%%
%% FRONTMATTER
%%%%%%%%%%%%%%%%%%%%%%%%%%%%%%%%%%%%%%%%%%%%

\begin{document}

\title{Computing astrophysical inferences with diffusive nested sampling}

\classification{<Replace this text with PACS numbers; choose from this list:
                \texttt{http://www.aip..org/pacs/index.html}>}
\keywords      {Astrophysics -- Bayesian Inference -- Nested Sampling}

\author{Brendon J. Brewer}{
  address={Department of Statistics, The University of Auckland\\
http://stat.auckland.ac.nz/\~{ }brewer/}
}


\begin{abstract}
Many standard problems in astronomy are best understood as inference problems.
In principle at least,
we should calculate the posterior distribution over a suitable hypothesis space,
rather than inventing ad-hoc procedures. I will describe two recent applications
of this idea: i) inferring the properties of active galactic nuclei (AGN)
from reverberation mapping data, and ii) producing catalogs (lists of objects
and their properties) from images of
crowded stellar fields. The first application allows us to measure the masses
of black holes and to infer the dynamical state of the matter surrounding them,
making the most of the large amount of telescope time required. The second
application has the potential to enable the study of faint objects that are not
even ``detected'' by standard algorithms.
To implement these models we used the Diffusive Nested Sampling algorithm,
which is a variant of Nested Sampling that was invented at MaxEnt 2009. The
algorithm naturally allows for unknown numbers of parameters since
``reversible jump'' MCMC can be used inside of it. However, there are
computational challenges involved when the posterior distribution is highly
compressed with respect to the prior distribution.
\end{abstract}

\maketitle

%%%%%%%%%%%%%%%%%%%%%%%%%%%%%%%%%%%%%%%%%%%%
%% MAINMATTER
%%%%%%%%%%%%%%%%%%%%%%%%%%%%%%%%%%%%%%%%%%%%

\section{Introduction}
The major goal of astronomy, as with most sciences, is to understand the
universe as well as we can while recognising that we will never have all of
the relevant information that we might want.



The first step is to construct a set of possible answers to the question of
interest, that is, the ``hypothesis space'', indexed by parameters $\pars$.
The prior distribution $p(\pars)$ describes a prior state of knowledge about
the parameters.

A joint prior is assigned on the hypothesis space of possible $\pars$ values
and possible data sets $\data$:
\begin{eqnarray}
p(\pars, \data) = p(\pars) p(\data | \pars) = p(\data)p(\pars | \data)
\end{eqnarray}
The marginal distribution $p(\pars)$, which is usually called the ``prior'',
describes initial uncertainty about the value of the parameters. When the
data $\data$ becomes known, our state of knowledge about $\pars$ changes
from the marginal $p(\pars)$ to the conditional distribution $p(\pars | \data)$
that corresponds to the observed data, usually called the ``posterior''.
Incidentally, Bayesian inference can be understood as a special case of
MaxEnt updating, starting from the joint prior $p(\pars, \data)$ and applying
the constraint $P(\data = \data_{\rm actual}) = 1$ \citep{caticha}. Note that
this is not a ``derivation'' of Bayesian updating as the rules of probability
already imply a commitment to update to $p(\pars | \data)$ once $\data$ becomes
known (what else could a conditional probability be?).

\section{Diffusive Nested Sampling}
DNS is essentially just the Metropolis algorithm. However, instead of the
target distribution being the posterior $p(\pars|\data)$, it is the mixture
of constrained priors.

\section{Probabilistic Catalogs}
The hypothesis space is the set of possible answers to the question
``how many objects are there, and what are their properties (positions, fluxes,
etc)''.


\section{Reverberation Mapping}
Many objects in the universe are too distant to resolve, so we cannot simply
``see'' what is going on. Rather we must infer what is going on. Active Galactic
Nuclei (AGN) are a good example.

\subsection{Models with known oversimplifications}
Computing the posterior $p(\pars | \data)$ would yield overconfident results
since the assumed model is known to be wrong. However certain parameters still
``exist'' (such as the black hole mass) and we want our posterior inferences
to be realistic.

We have been using the following subjective procedure. Later we give two
interpretations of this procedure which may lead the way to less subjective
approaches.
Using DNS we explore the mixture of constrained priors.
\begin{eqnarray}
p_{\rm DNS}(\theta) &=& \frac{1}{n}\sum_{i=1}^n
\frac{p(\theta)\mathds{1}\left[L(\theta) > L^*_i\right]}{X_i}
\end{eqnarray}
where $X_i \approx e^{-i}$ by construction.

%%%%%%%%%%%%%%%%%%%%%%%%%%%%%%%%%%%%%%%%%%%%%%%%
%% BACKMATTER
%%%%%%%%%%%%%%%%%%%%%%%%%%%%%%%%%%%%%%%%%%%%%%%%

\begin{theacknowledgments}
Acknowledgements.
\end{theacknowledgments}

%%%%%%%%%%%%%%%%%%%%%%%%%%%%%%%%%%%%%%%%%%%%%%%%
%% The bibliography can be prepared using the BibTeX program or
%% manually.
%%
%% The code below assumes that BibTeX is used.  If the bibliography is
%% produced without BibTeX comment out the following lines and see the
%% aipguide.pdf for further information.
%%
%% For your convenience a manually coded example is appended
%% after the \end{document}
%%%%%%%%%%%%%%%%%%%%%%%%%%%%%%%%%%%%%%%%%%%%%%%%

%%%%%%%%%%%%%%%%%%%%%%%%%%%%%%%%%%%%%%%%%%%%%%%%
%% You may have to change the BibTeX style below, depending on your
%% setup or preferences.
%%
%%
%% For The AIP proceedings layouts use either
%%%%%%%%%%%%%%%%%%%%%%%%%%%%%%%%%%%%%%%%%%%%

\bibliographystyle{aipproc}   % if natbib is available
%\bibliographystyle{aipprocl} % if natbib is missing

%%%%%%%%%%%%%%%%%%%%%%%%%%%%%%%%%%%%%%%%%%%
%% You probably want to use your own bibtex database here
%%%%%%%%%%%%%%%%%%%%%%%%%%%%%%%%%%%%%%%%%%%
%\bibliography{sample}

%%%%%%%%%%%%%%%%%%%%%%%%%%%%%%%%%%%%%%%%%%%%
%%% Just a reminder that you may have to run bibtex
%%% All of it up to \end{document} can be removed
%%% if you don't like the warning.
%%%%%%%%%%%%%%%%%%%%%%%%%%%%%%%%%%%%%%%%%%%%
%\IfFileExists{\jobname.bbl}{}
% {\typeout{}
%  \typeout{******************************************}
%  \typeout{** Please run "bibtex \jobname" to optain}
%  \typeout{** the bibliography and then re-run LaTeX}
%  \typeout{** twice to fix the references!}
%  \typeout{******************************************}
%  \typeout{}
% }

\end{document}

%%%%%%%%%%%%%%%%%%%%%%%%%%%%%%%%%%%%%%%%%%%
%% The following lines show an example how to produce a bibliography
%% without the help of the BibTeX program. This could be used instead
%% of the above.
%%%%%%%%%%%%%%%%%%%%%%%%%%%%%%%%%%%%%%%%%%%

\begin{thebibliography}{9}

\bibitem{Brown2000}
M.~P. Brown,  and K.~Austin, \emph{The New Physique}, Publisher Name,
  Publisher City, 2000, pp. 212--213.

\bibitem{BrownAustin:2000}
M.~P. Brown,  and K.~Austin, \emph{Appl. Phys. Letters} \textbf{85},
  2503--2504 (2000).

\bibitem{Wang}
R.~Wang, ``Title of Chapter,'' in \emph{Classic Physiques}, edited by
  R.~B. Hamil, Publisher Name, Publisher City, 2000, pp. 212--213.

\bibitem{SJ:1999}
C.~D.~Smith and E.~F.~Jones,  ``Load-Cycling in Cubic Press,'' in
  \emph{Shock Compression of Condensed Matter-1999}, edited by M.~D.~F. et~al.,
  AIP Conference Proceedings 505, American Institute of Physics, New York,
  1999, pp. 651--654.

\end{thebibliography}

\endinput
%%
%% End of file `template-8s.tex'.
