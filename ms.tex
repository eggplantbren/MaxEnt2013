\documentclass[letterpaper, 11pt]{article}
\usepackage{graphicx}
\usepackage{natbib}
\usepackage[left=3cm,top=3cm,right=3cm]{geometry}

\renewcommand{\topfraction}{0.85}
\renewcommand{\textfraction}{0.1}
\parindent=0cm

\title{Document Title}
\author{Brendon J. Brewer}


\begin{document}
\maketitle

\abstract{Many standard problems in astronomy are best understood as inference problems.
In principle at least,
we should calculate the posterior distribution over a suitable hypothesis space,
rather than inventing ad-hoc procedures. I will describe two recent applications
of this idea: i) inferring the properties of active galactic nuclei (AGN)
from reverberation mapping data, and ii) producing catalogs (lists of objects
and their properties) from images of
crowded stellar fields. The first application allows us to measure the masses
of black holes and to infer the dynamical state of the matter surrounding them,
making the most of the large amount of telescope time required. The second
application has the potential to enable the study of faint objects that are not
even ``detected'' by standard algorithms.
To implement these models we used the Diffusive Nested Sampling algorithm,
which is a variant of Nested Sampling that was invented at MaxEnt 2009. The
algorithm naturally allows for unknown numbers of parameters since
``reversible jump'' MCMC can be used inside of it. However, there are
computational challenges involved when the posterior distribution is highly
compressed with respect to the prior distribution.}

\end{document}

