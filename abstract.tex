%%%%%  LaTeX template for submission of abstracts to  MaxEnt 2013         
%%%%%                                                         
%%%%% Use this template to generate a pdf version of your abstract for submission. This is 
%%%%% most easily accomplished using a LaTeX compiler that outputs directly to pdf.
%%%%%                                      
%%%%% Your abstract may be no more that one page in length and that all text and figures must                                                      
%%%%% fit in a 16.0 cm x 23.0 cm frame. 
%%%%%                                          
%%%%% When you submit the pdf version of your abstract on the conference web site
%%%%% you will be ask if your are requesting an Oral presentation or a Poster.  You will                             
%%%%% also be ask it this is a student presentation.                                          
%%%%%                                                                
                                                      
%%%%% Do not change the text size in the line below.
\documentclass[letterpaper,12pt]{article}
%\documentclass[final,onecolumn]{aipproc}
%\layoutstyle{6x9}

%%%%%  If you chose not to include the optional figure then you can delete the line below.
\usepackage{graphicx}
\usepackage{amsmath}
\usepackage{amssymb}

%\usepackage{empheq}   %boxes around equations
%\usepackage{multicol}        % used for the two-column index
%\usepackage[bottom]{footmisc}	% places footnotes at page bottom
%\usepackage{epstopdf} 	%essential with TeXShop on Mac
%\usepackage{longtable} 	%long tables
%\usepackage{esint}  	%multiple integral fonts
%\usepackage{stmaryrd} 	%floor, ceiling symbol font

%\hyphenation{MaxEnt}

%%%%% Page size and layout commands. Do not change lines below below.
\textwidth16cm
\textheight23cm
\topmargin-2.0cm
\oddsidemargin0cm
\parindent0pt
%\pagestyle{empty}
%%%%% End of Page size and layout commands.


\begin{document}


%%%%%  The title of your contribution in all capital letters.
\title{COMPUTING ASTROPHYSICAL INFERENCES WITH DIFFUSIVE NESTED SAMPLING}

%%%%%  Author's name (underline presenter's author)            
\author{\underline{Brendon J. Brewer}$^{1}$\\
        (1) Department of Statistics, The University of Auckland\\
	\\
        {\tt http://stat.auckland.ac.nz/\~{ }brewer/}
       }

\date{}
\maketitle

%%%%%               Abstract begins here                      
\begin{abstract}
\noindent

Many standard problems in astronomy are best understood as inference problems.
In principle at least,
we should calculate the posterior distribution over a suitable hypothesis space,
rather than inventing ad-hoc procedures. I will describe two recent applications
of this idea: i) inferring the properties of active galactic nuclei (AGN)
from reverberation mapping data, and ii) producing catalogs (lists of objects
and their properties) from images of
crowded stellar fields. The first application allows us to measure the masses
of black holes and to infer the dynamical state of the matter surrounding them,
making the most of the large amount of telescope time required. The second
application has the potential to enable the study of faint objects that are not
even ``detected'' by standard algorithms.
To implement these models we used the Diffusive Nested Sampling algorithm,
which is a variant of Nested Sampling that was invented at MaxEnt 2009. The
algorithm naturally allows for unknown numbers of parameters since
``reversible jump'' MCMC can be used inside of it. However, there are
computational challenges involved when the posterior distribution is highly
compressed with respect to the prior distribution.


\vspace{1cm}
\medskip\noindent{References: }

[1] {\it Diffusive Nested Sampling}\\
Brendon J. Brewer, Livia B. P{\'a}rtay, Gabor Cs{\'a}nyi,
Statistics and Computing, 2011, 21, 4, 649-656.\\

[2] {\it Geometric and Dynamical Models of Reverberation Mapping Data}\\
Anna Pancoast, Brendon J. Brewer, Tommaso Treu, 2011, ApJ, 730, 139.\\

[3] {\it Probabilistic Catalogs for Crowded Stellar Fields}\\
Brendon J. Brewer, Daniel Foreman-Mackey, David W. Hogg, 2013, AJ, 146, 7.\\
%%%%%        END of References



%%%%%       OPTIONAL: Key Words (delete if not used)   
\medskip\noindent{Key Words: }
{Astrophysics, Bayesian Inference, Markov Chain Monte Carlo}
%%%%%        END of Key Words

\end{abstract}
\thispagestyle{empty}
\end{document}

\endinput
