%%%%%  LaTeX template for submission of abstracts to  MaxEnt 2013         
%%%%%                                                         
%%%%% Use this template to generate a pdf version of your abstract for submission. This is 
%%%%% most easily accomplished using a LaTeX compiler that outputs directly to pdf.
%%%%%                                      
%%%%% Your abstract may be no more that one page in length and that all text and figures must                                                      
%%%%% fit in a 16.0 cm x 23.0 cm frame. 
%%%%%                                          
%%%%% When you submit the pdf version of your abstract on the conference web site
%%%%% you will be ask if your are requesting an Oral presentation or a Poster.  You will                             
%%%%% also be ask it this is a student presentation.                                          
%%%%%                                                                
                                                      
%%%%% Do not change the text size in the line below.
\documentclass[letterpaper,12pt]{article}
%\documentclass[final,onecolumn]{aipproc}
%\layoutstyle{6x9}

%%%%%  If you chose not to include the optional figure then you can delete the line below.
\usepackage{graphicx}
\usepackage{amsmath}
\usepackage{amssymb}

%\usepackage{empheq}   %boxes around equations
%\usepackage{multicol}        % used for the two-column index
%\usepackage[bottom]{footmisc}	% places footnotes at page bottom
%\usepackage{epstopdf} 	%essential with TeXShop on Mac
%\usepackage{longtable} 	%long tables
%\usepackage{esint}  	%multiple integral fonts
%\usepackage{stmaryrd} 	%floor, ceiling symbol font

%\hyphenation{MaxEnt}

%%%%% Page size and layout commands. Do not change lines below below.
\textwidth16cm
\textheight23cm
\topmargin-2.0cm
\oddsidemargin0cm
\parindent0pt
%\pagestyle{empty}
%%%%% End of Page size and layout commands.


\begin{document}


%%%%%  The title of your contribution in all capital letters.
\title{ASTRONOMICAL INFERENCE WITH DIFFUSIVE NESTED SAMPLING}

%%%%%  Author's name (underline presenter's author)            
\author{\underline{Brendon J. Brewer}$^{1}$\\
        (1) Department of Statistics, The University of Auckland\\
	\\
        {\tt http://stat.auckland.ac.nz/\~{ }brewer/}
       }

\date{}
\maketitle

%%%%%               Abstract begins here                      
\begin{abstract}
\noindent

Many standard problems in astronomy (e.g. the processing of images to produce
``catalogs'') are best understood as inference problems. In principle at least,
we should calculate the posterior distribution over a suitable hypothesis space,
rather than inventing ad-hoc procedures.

I will describe the Diffusive Nested Sampling method, which is a variant of
Nested Sampling that retains the flexibility of Markov Chain Monte Carlo
exploration of the hypothesis space.

Talk about the StarField project
Talk about Reverberation Mapping
Talk about how Nested Sampling is hard when you need to compress so far
Talk about reversible jump

%%%%%       OPTIONAL: Figure (delete if not used)                                                 
%%%%%       Do not add a figure caption.
%%%%%       The conference program will not be printed in colour so colour figures should not be used.
%%%%%       The example figure is in pdf format.     

%\begin{figure}[h]
%\begin{center}
%\includegraphics[width=70mm]{roo}
%\end{center}
%%\caption{blah blah.}
%%\label{fig:roo}
%\end{figure}
%%%%%        End of Figure


%%%%%       OPTIONAL: References (delete  if not used)   
%%%%%       Separate each reference with a blank line.                                                                
\medskip\noindent{References: }

[1] A. Dent et al. Phys. Rev. {\bf E99}, 99998 (2009).

[2] Z. Beeblebrox et al. Phys. Rev. {\bf E99}, 99999 (2009).
%%%%%        END of References



%%%%%       OPTIONAL: Key Words (delete if not used)   
\medskip\noindent{Key Words: }
{Bayesian Inference, Markov Chain Monte Carlo}
%%%%%        END of Key Words

\end{abstract}
\thispagestyle{empty}
\end{document}

\endinput
